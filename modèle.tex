%préambule, permet de faire les réglages

%utilisation du modèle "article"
\documentclass[11pt, a4paper]{article}

%Dimension de la page
%\usepackage[cm]{fullpage}
\usepackage[vmargin=2cm,hmargin=2cm]{geometry}

%Remplissage lipsum
\usepackage{lipsum} 

%permet l'utilisation des caractères accentués
\usepackage[utf8x]{inputenc}
\usepackage[T1]{fontenc}

%pour insérer des images 
\usepackage{graphicx}

%pour les formules mathématiques
\usepackage{amsmath, amsfonts}
\usepackage{amssymb, amsthm}

%typographie française
\usepackage[french]{babel}

%pour dessiner avec Latex
\usepackage{tikz}

%pour écrire les unités
\usepackage{siunitx}

%pour les en-têtes et pieds de page personnalisés (à gauche L, au centre C et à droite R)
\usepackage{fancyhdr}
\pagestyle{fancy}
\fancyhead[L]{\includegraphics[scale=0.2]{insa.pdf} } %pour avoir le logo de l'insa à gauche (L), le fichier insa.pdf doit se trouver dans le même dossier que le fichier tex. 
\fancyhead[C]{Titre du TP  } % peut permettre d'indiquer un titre court au centre (C)
\fancyhead[R]{Auteurs } % peut permettre d'indiquer le nom de l'auteur (R)
\renewcommand\headrulewidth{2pt}
\fancyfoot[L]{Poste : } % pour insérer le numéro de poste
\fancyfoot[C]{Page \thepage} % pour insérer le numéro de la page
\fancyfoot[R]{le : } % pour insérer la date  à droite (R)

%si l'on souhaite retirer l'une ou l'autre des parties des pieds de page ou en-tête, il suffit de "commenter" la ligne en ajoutant un signe pourcentage devant cette ligne. 

\usepackage{fancybox}% pour encadrer quelques mots à l'aide de la commande \fbox

\title{Modèle pour la rédaction de rapports scientifiques}
\author{Auteurs}
\date{Date}


\begin{document}
\maketitle
\thispagestyle{fancy}


Introduction

\section{Titre de section de niveau 1}
\subsection{Titre de sous-section de niveau 2}
\subsubsection{Titre de sous-sous-section de niveau 3}

Contenu

\bigskip


Faire une liste non numérotée :
\begin{itemize}
\item premier item ;
\item deuxième item.
\end{itemize}

\bigskip

Faire une liste numérotée. 
\begin{enumerate}
\item Premier item. 
\item Deuxième item.
\end{enumerate}

\bigskip

Pour introduire une figure avec une légende et un renvoi possible dans le texte à la figure \ref{insa}. 

\begin{figure}[h] %option h permet de mettre l'image à l'endroit indiqué
\centering % permet de centrer l'image 
\includegraphics[width=0.75\textwidth]{insa} % indiquer le nom du fichier de figure qui doit être enregistré dans le même dossier que le fichier tex, [scale=1] permet de régler la taille de la figure
\caption{Légende de la figure : exemple avec le logo INSA} % permet de donner un titre à la figure
\label{insa}%donne le nom de référence de la figure qui peut être rappelée dans le texte à l'aide de \ref{insa}. Attention, il faut compiler deux fois pour que la numérotation des figures soit mise à jour dans le fichier pdf !
\end{figure}

Exemple texte


\lipsum % permet de générer un texte appelé lipsum pour la mise en page. Donc ligne à commenter ou retirer lors de l'utilisation ! 

\end{document}